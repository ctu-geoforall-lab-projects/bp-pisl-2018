\chapter{Introduction}
\label{1-introduction}

Cloud computing - the delivery of on-demand computing resources over a
network (internet or intranet) - is on the rise. Computing resources
may range from relatively simple ones such as text editors to very
complex software. They all share the same basic concept – the client
only needs to provide input data, the process itself runs on a server
and then returns the corresponding output data to the client.

Using cloud computing offers a way to gain new capabilities without
investing in new hardware or software. It allows users to perform
complex tasks without having to deal with technical details of the
process. It ensures better collaboration by allowing remote teams to
use the same software. And as all resources are maintained by the
service provider, usually the latest version of the software is
available.

For the above mentioned and other reasons, cloud computing is
frequently used in the field of geoinformation technology and GIS. It
is most useful when it is clear to the client what they need to do and
how to do it and they only lack the tools to perform the task. For
example, the ESRI’s ArcGIS Online platform offers a wide selection of
tools to choose from. These can be “rented” by the client and the task
is then performed on the server. That way, the client gets the desired
data without having to purchase (and learn to use) any desktop GIS
application.

While the benefits of cloud computing are apparent, there is also
another aspect of it that may not get as much attention – the
management of the data produced by such processes. The amount of data
collected and used is increasing rapidly and so is the number of
processes that are being shared on a network. Naturally, the amount of
data derived from such services is growing, too.

Since one of the primary motivation behind cloud computing is to make
the service easy to use for the client, managing output data should
also be carried out by the provider of the service in such a way that
is convenient for the client – it needs to be effective,
well-organized and easy to access.

Relational databases are one way to do so. They are designed and
organized especially for rapid search and retrieval in what can be a
large amount of data. Database-management system (DBMS) is a tool for
managing and interacting with databases.
	
DBMSs have several major advantages over the traditional system where
data is stored in files. Unlike the file management system, there can
be more users accessing the same data concurrently without corrupting
the data. Indexing speeds up the data retrieval operations. There is a
standardized database language to use for queries. There are
mechanisms such as data normalization that can be used to avoid
duplicity of data and save storage space.
	
Being aware of these advantages, the main goal of this thesis is to
design an extension, written in Python, that implements database
storage for output data derived from geoprocessing services run within
the PyWPS framework. They typically produce geographical data, so
appropriate software must be used that is capable of dealing with this
type of data. In this thesis, PostgreSQL is used together with its
spatial database extender PostGIS.
