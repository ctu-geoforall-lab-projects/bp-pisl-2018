\chapter{Úvod}
\label{1-uvod}

The amount of data that is being collected and that people work with is increasing rapidly. That is due to the new technologies and instruments for collecting data that are being developed and used. While it provides data for more thorough and precise analysis than ever before, it also makes the data organization a much more difficult and complex task for which new methods had to be developed.

Relational databases are such a method. They are designed and organized especially for rapid search and retrieval in what can be a large amount of data. The database-management systems (DBMS) are a tool for managing and interacting with databases. DBMS have several major advantages over the traditional system where data is stored in files. Unlike the file management system, there can be more users accessing the same data concurrently without corrupting the data. Indexing speeds up the data retrieval operations. There is a standardized database language to use for queries. There are mechanisms such as data normalization that can be used to avoid duplicity of data and save storage space.
	
Being aware of these advantages, the main goal of this thesis is to explore the options of integrating an extension to the PyWPS framework that would allow the output data to be stored in a database. PyWPS is an implementation of the Web Processing Service standard from the Open Geospatial Consortium. The Web Processing Service (WPS) Interface Standard provides rules for standardizing inputs and outputs (requests and responses) for geospatial processing services. Written in Python, PyWPS enables integration, publishing and execution of Python processes via the WPS standard.
	
As of now, PyWPS is saving output data as files that are uploaded on the server. The client is then given a URL link from which they can download the file. If there was an option for the data to be stored in a database it could significantly improve the efficiency of the data transfer. 
 
Also, the issue of actually implementing this extension to the PyWPS framework will be addressed in this thesis.
