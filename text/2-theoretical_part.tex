\chapter{Theory}
\label{2-teorie}

\section{Web Processing Services}

The Web Processing Service Standard (WPS) is an Open Geospatial Consortium (OGC) standard that provides rules for publishing and executing processes on the web. „The standard also defines how a client can request the execution of a process, and how the output from the process is handled. It defines an interface that facilitates the publishing of geospatial processes and clients’ discovery of and binding to those processes. The data required by the WPS can be delivered across a network or they can be available at the server.“ \cite{wpsstandard}

WPS uses HTTP and XML (eXtensible Markup Language) for describing processes and data transfer. The first version 0.4.0 was released in 2005. Despite a new version 2.0.0 having been released in 2015, version 1.0.0 remains most widely used.

A process is essentially a function p that returns an output Y for each input X:\\
\centerline{p: X -> Y}

In case of WPS, a process is a geospatial operation, calculation or a model of any complexity. It may require one or more input arguments and always yields one or more outputs. If there are no input arguments, the process is either generating constant or random values. While WPS was designed for geospatial data, it is not restricted to them.

\begin{figure}[H] \centering
      \includegraphics[width=400pt]{./pictures/wps_conceptual_model.png}
      \caption[QGIS logo]{WPS conceptual model (source: WPS)}
      \label{fig:WPS}
  \end{figure}


There are two basic capabilities of a WPS server. It provides the process and retrieves the process description and it controls and monitors processing jobs. Job is an instance of a process – it is an object created for a particular process execution. Job control is the ability to execute, dismiss or delete a job.

\subsection{Process execution}


There are two ways in which a process can be executed. If the complexity of the process is lower, the connection is stable and the completion time is relatively short (for instance, the Apache2 server uses 60 seconds as a default value for ConnectionTimeout\cite{apache})  the execution is run synchronously. After the execute request is sent by the client, the WPS server starts executing the process while the client remains connected to the server for the entire time of the execution. Only when the process has finished and the output has been delivered, the connection terminates.

The asynchronous execution, on the other hand, better suits for complex processes that are expected to take longer time to finish. When the client requests process excecution, the server responds with a status information message that confirms the request has been accepted and the process will be run. The message also includes a unique processing job identifier. Then, the connection is interrupted. At this time, the client can send a GetStatus request with the job identifier to get information on how the process has progressed. Once the process has finished, the client can access the output data using a GetResult request with the job identifier. The asynchronous execution can be also useful in case of unstable connection that prevents synchronous execution to successfully run.

\subsection{Available operations}

\textbf{GetCapabilities.} A mandatory operation for any OGC Web service. For WPS, it is one of the three basic operations that are available for any process. There is no input parameter required. Once the request is sent, the server returns a document that includes information about the service provider, a list of operations available for a WPS server and a list of all the processes offered by the service.

\noindent \textbf{DescribeProcess.} Another basic operation. The only required input is the name (or a list of names) of the process to be described. If the process offers description in multiple languages, a language parameter can be added to the request. It returns a document with characteristics of the process and a description of input parameters. It also describes the output format of the process.

\noindent \textbf{Execute.} The key operation for any OGC WPS service. It allows processes that are implemented by a server to be run. It requires input parameter values as specified by the service. These can be either values (numerical or other) that are included directly in the request or references to other recources that are accessible as local files or through the web. Analogically, outputs of the execute operation can be included in the response XML document or stored locally or on the web. In case of anyschronous operation, the response contains a unique JobID that the client uses to enquire about the process status and results.

\noindent \textbf{GetStatus.} An operation only used for asynchronous processes. After the Execute request has been accepted, the client can use this operation to query status information of the processing job, using a JobID returned by the Execute response. This operation is only available in version 2.0.0.

\noindent \textbf{GetResult.} The final operation used for asynchronous processes that is to be used after the GetStatus operation reports the process has finished. The input parameter is the unique JobID. Then, GetResult fetches the results of the process as described in the Execute section. This operation is only available in version 2.0.0.

\noindent \textbf{Dismiss.} Allows the client to let the server know the result of the process is no longer wanted. In such case, all the resources corresponding to the JobID sent in the Dismiss request may be deleted. If the process is still running, it may be cancelled. This operation is only available in version 2.0.0.

\subsection{OGC WPS Implementations}

%% ML: itemize -> subsection

\subsubsection{PyWPS}
\begin{figure}[H] \centering
      \includegraphics[width=200pt]{./pictures/pywps.png}
      \caption[PyWPS logo]{PyWPS (source:
\href{http://pywps.org/images/pywps.png}{PyWPS})}
      \label{fig:PyWPS}
  \end{figure}

PyWPS is a server side implementation of the OGC Web Processing Service (OGC WPS) standards 1.0.0. It is written in the Python programming language, it runs on Python 2.7, 3.3 or higher and it is tested and developed on Linux. It uses a ConfigParser format for configuration files. It supports a variety of geospatial software and tools such as GRASS GIS, R Project or the GDAL library. Synchronous and asynchronuous invocations are supported. As for request encoding, two options are available - key-value pairs (using HTTP-GET) or XML payload (using HTTP-POST). Every process that is to be deployed on the server is defined as a class and has several mandatory parameters. The key parameter called "handler" gets invoked every time there is an incoming request, it accepts the request and returns a response. 

In 2016, it upgraded from PyWPS 3 to PyWPS 4. Some of the more significant changes include every input being considered a list of inputs and all inputs having file, data and stream attributes. These attributes allow better manipulation with data. 
	
As PyWPS only supports  version 1.0.0 of the OGC WPS standards, it does not support operations implemented in version 2.0.0. These operations are GetStatus, GetResult and Dismiss. Support for version 2.0.0 is currently being planned.\cite{pywps}
 

\subsubsection{ZOO-Project}
\begin{figure}[H] \centering
      \includegraphics[width=200pt]{./pictures/zoo.png}
      \caption[ZOO-Project logo]{ZOO-Project (source: ZOO-Project)}
      \label{fig:ZOO-Project}
  \end{figure}

ZOO-project is an open source WPS platform that consists of several components. The core processing engine, ZOO-Kernel, is a WPS server written in C that implements WPS standards 1.0.0 and 2.0.0.\cite{zoo} A significant advantage over other WPS implementations is that it is written as a polyglot, i.e. in a valid form of multiple programming languages, which performs the same operations independent of the programming language used to compile or interpret it.\cite{polyglot} It runs on Mac OS, Linux and Microsoft Windows operating systems. It uses a ConfigParser styled configuration file.

ZOO-services offers a rich collection of ready-to-use services that are built on open source libraries such as GDAL or GRASS GIS. 

ZOO-API is a server-side library written in JavaScript for creation and chaining services. These services can be written in one of five programming languages that are supported. It also offers easy conversion of vector formats. 

ZOO-Client is a simple client-side JavaScript API for interacting with WPS from web applications. It allows to build WPS requests and send them to a WPS server. It also provides functions to easily parse the output XML responses.


\subsubsection{52°North WPS}
\begin{figure}[H] \centering
      \includegraphics[width=150pt]{./pictures/52n.png}
      \caption[52° North logo]{52 North (source: 52° North)}
      \label{fig:52 North}
  \end{figure}

52°North WPS is a part of the 52°North open source software initiative. Located in Germany, their aim is to foster innovation in the field of geoinformatics through a collaborative process. As a part of this inititiave, the 52°North WPS is an implementation of the OGC WPS standard (version 1.0.0). It is written in the Java programming language. It can be run under Linux or Microsoft Windows operating systems. 

As for the WPS invocation methods, it supports both synchronous and asynchronous invocation, HTTP-GET and HTTP-POST. As for the WPS datatypes, it supports GeoTIFF, Shapefile, KML, WKT and others.\cite{north} Configuration is based on an XML file.
  

\subsubsection{ESRI Web Processing}

\begin{figure}[H] \centering
      \includegraphics[width=200pt]{./pictures/esri.png}
      \caption[ESRI logo]{ESRI (source: ESRI)}
      \label{fig:ESRI}
  \end{figure}

ESRI is an international company oriented on desktop and mobile GIS software, geodatabases and web GIS. Founded in 1969, it is the leading company in the global GIS market. It offers a variety of GIS products, including ArcGIS for Desktop, ArcGIS Online or ArcGIS for Mobile. 

It allows services created within the ArcGIS software to be published and shared online on another ESRI's platform, ArcGIS Server. On ArcGIS Server, these services can be stored and accessed by other users. They can be also implemented alongside with maps, which can also be created in other ArcGIS software and then published online, into online web applications. Users can also take advantage of the Web App Builder feature and of many templates that simplify creating applications.

By default, ESRI software does not follow WPS standards. However, when publishing a geoprocessing service in ArcGIS Desktop, there is a possibility to enable the WPS capability. Then, the service published is compliant with the OGC WPS 1.0.0 specifications.\cite{arcgiswps}

\section{Spatial Databases}

Spatial database (or a geodatabase) is a database optimized for storing and querying data related to objects in geometric space, such as points, lines or polygons. They require additional functionality for processing spatial data effectivelly. Usually, special data types such as geometry or feature are added along standard data types. In addition to typical SQL queries such as SELECT statements, spatial databases can perform a wide variety of spatial operations. These spatial operations include (but are not restricted to) computing line length, polygon area, distance between geometries, etc.

The International Organization for Standardization and OGC specifies a Simple Features standard that is divided into two parts. The first one defines a general model for two-dimensional geometries. It also deals with spatial reference systems. The second part defines an implementation using SQL. This second part is implemented to varying extent in most of spatial databases and extensions.

Here I list the best known and most widely used spatial database systems, however, there are many more. In fact, most of the major database systems support spatial data, including Microsoft SQL Server or MySQL (and its community developed branch MariaDB). Also, there is a number of database systems especially designed and developed as spatial databases (e.g. SpatialDB, SpaceBase, MapD).

\subsection{PostGIS}
\begin{figure}[H] \centering
      \includegraphics[width=190pt]{./pictures/postgis.png}
      \caption[PostGIS logo]{PostGIS (source: PostGIS)}
      \label{fig:PostGIS}
  \end{figure}
  
PostGIS is an open-source spatial database extension for PostgreSQL. PostgreSQL is a widely-used object-relational database management system. PostGIS adds support for geographic objects according to the OGC Simple Features for SQL specification. 

PostgreSQL and PostGIS use the client/server architecture. When a client makes a request (typically an SQL statement), there is a server that accepts and evaluates it. The server itself is then responsible for updating (or, generally, changing in any way) the database file. Two of the most significant advantages of PostgreSQL over other DBMS are its high standard compliance and extensibility. It allows "stored procedures" to be created and saved to simplify complex operations that are frequently repeated. 

In a study that compared PostGIS with Oracle Spatial, PostGIS  was found to perform faster when accessing and querying data.\cite{postgis} 

	
\subsection{Oracle Spatial and Graph}

\begin{figure}[H] \centering
      \includegraphics[width=200pt]{./pictures/oracle.png}
      \caption[Oracle logo]{Oracle (source: Oracle)}
      \label{fig:Oracle}
  \end{figure}
  
  
The Oracle Spatial and Graph is an extension of the Oracle Database that allows managing geographic data in a native type within an Oracle database. It is also a client-server service but, unlike PostGIS, it is proprietary. Spatial features extend Oracle Locator, a standard feature of the Oracle Database distribution. Oracle Locator provides basic functions and services in Oracle Spatial but it lacks more advanced functions. Oracle Spatial and Graph supports large-scale geographic information systems, provides spatial web services and generally is designed for complex spatial data management and analysis.

\subsection{SpatiaLite}

\begin{figure}[H] \centering
      \includegraphics[width=150pt]{./pictures/spatialite.png}
      \caption[SpatialLite logo]{SpatialLite (source: SpatialLite)}
      \label{fig:SpatialLite}
  \end{figure}
  
SpatiaLite is a lightweight library that extends the SQLite database management system so it provides support for spatial data. Unlike Oracle and PostgreSQL, SQLite is designed as a file-based database. It means that, unlike server-based databases, by using SQL expressions a user updates the file directly. Therefore, a file-based database must be stored in the local file system. As a consequence, SQLite is very fast and efficient for standard operations but is not optimized for multi-user applications or for large-scale operations. SQLite (and SpatiaLite) is open-source.

\subsection{ArcSDE \& Geodatabase (ESRI)}

\begin{figure}[H] \centering
      \includegraphics[width=170pt]{./pictures/arcsde.png}
      \caption[ArcSDE logo]{ArcSDE (source: ArcSDE)}
      \label{fig:ArcSDE}
  \end{figure}
	
ESRI is a supplier of a wide range of GIS software, including web GIS services, desktop and mobile applications and geodatabase management systems. Data used, produced or derived from ESRI software is stored in geodatabases, that are defined by ESRI as "a collection of geographic datasets of various types held in a common file system folder, a Microsoft Access database, or a multiuser relational DBMS."\cite{esridef}  As the definition suggests, ESRI distinguishes between three conceptually different types of geodatabases, depending on the scale of the data, requirements of the client, operating system and other factors.

In its simplest form, an ESRI geodatabase can be a collection of GIS data stored as files in a folder within the standard file system. This is called file geodatabase and is intended for single users and small workgroups. Personal geodatabase is another type of geodatabase, also designed to be used by individuals or small workgroups. It uses Microsoft Access database management system and therefore is only available on Microsoft Windows operating system. 

The final and most complex type of geodatabase is the enterprise geodatabase. Unlike the previous two, it is designed to be used by multiple users simultaneously and for large datasets. It works with various DBMS storage models using ArcSDE. 

ArcSDE is a proprietary technology for managing and accessing spatial data that supports multiple DBMS: IBM DB2, IBM Informix, Microsoft SQL Server, Oracle, and PostgreSQL.\cite{arcsdedoc}  It also supports the corresponding standards. ArcSDE is built with the client/server architecture. A client first sends a request to the server. Then, the server receives the request, generates results, and delivers them to the client.

\begin{figure}[H] \centering
      \includegraphics[width=200pt]{./pictures/arcsdeobr.png}
      \caption[ArcSDE logic]{ArcSDE logic(source: ArcSDE)}
      \label{fig:ArcSDE logic}
  \end{figure}

The client doesn't need any knowledge of particulars of any of the DBMS. Another significant advantage is that ArcSDE allows datasets to be available to multiple users for viewing, querying or editing at the same time.\cite{esritypes}

\section{Background research}


PyWPS is only one of the many implementations of the WPS standard, each of them approaching the problem of storing output data differently. From those listed above, the ZOO-project provides the "ZOO-kernel optional database support" \cite{zoodb} that is most similar to what the author of this thesis is aiming to create. There is an optional section in the configuration file that allows to configure the connection. The section has six elements (dbname, port, user, host, type and schema) that are used to generate a connection string that is passed to the GDAL library that connects to the database.\cite{zoodbsec} 


When using ArcGIS Server (ESRI software for executing processes on a server), ArcSDE handles storing data in a database. For more information about the ArcSDE technology, refer to the section 2.2.4 - ArcSDE \& Geodatabase (ESRI).

As for the 52°North Web Processing Service, it stores output data as web accessible resources and provides the consumer with an URL. Depending on the type of the output data, it may be stored directly as WMS, WFS or WCS layers.\cite{north} It does not, however, include the option to store output data in a database.