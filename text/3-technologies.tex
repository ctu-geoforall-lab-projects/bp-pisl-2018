\chapter{Technology}
\label{3-technologie}


\section{Python}

\begin{figure}[H] \centering
      \includegraphics[width=180pt]{./pictures/python-logo-master-v3-TM.png}
      \caption[Python logo]{Python logo (source:
\href{https://www.python.org/static/community_logos/python-logo-master-v3-TM.png}{Python.org})}
      \label{fig:python}
  \end{figure}

  Python is a high-level programming language that fully supports
  object-oriented and structured programming. Developed in the late
  1980s, the first version 0.9.0 was released in 1991. In 2008, Python
  3.0 was released. Currently, the most up-to-date version available
  is 3.6.\cite{diveintopython}

  It was designed as a syntactically simple language, using whitespace
  intendantion instead of brackets and English words rather than
  punctuation. It is a dynamically-typed language, which means it is
  not neccessary to specify a data-type when defining a variable. For
  its simplicity and readability, Python is often considered a good
  first programming language to learn.

  One of the key advantages of Python is its high extensibility. It
  provides large standard libraries and also an extensive number of
  other modules, packages and libraries, so most of the common
  programming tasks are already solved, scripted and made available.



\section{GitHub}

\begin{figure}[H] \centering
      \includegraphics[width=170pt]{./pictures/github.png}
      \caption[GitHub logo]{GitHub logo (source:
\href{GitHub}{GitHub.com})}
      \label{fig:GitHub}
  \end{figure}

  GitHub is a web-based Git repository hosting service with a
  graphical interface. Git is an open-source version control system
  for tracking changes in text files, typically used for source code
  management.\cite{git} On top of the standard Git functionality,
  GitHub provides a number of its own features, including forking
  (copying a repository), pull requests, or bug tracking. GitHub also
  offers a desktop application.

\section{Geospatial Data Abstraction Library} \label{gdal}

\begin{figure}[H] \centering
      \includegraphics[width=130pt]{./pictures/gdal.png}
      \caption[GDAL logo]{GDAL logo (source:{\cite{gdal}})}
      \label{fig:GDAL}
  \end{figure}

  %% ML: v soucasnosti se jiz pouziva pouze zkratka GDAL, opravil jsem to
  %% ML: dvakrat geospatial ve jedne vete, zkuste prepsat
  %% JP: opraveno, jednou to tam bylo navic
  Geospatial Data Abstraction Library (GDAL) is the most widely
  used data acces library for raster and vector geospatial
  data formats. It is released under an X/MIT style Open Source
  license by the Open Source Geospatial Foundation and it is written
  in C++ and C programming languages. As for operating systems, it can
  run under Linux, Solaris, Mac OS X and Microsoft Windows.\cite{gdal}

  The first version was released by Frank Warmerdam in 2000 and the
  last stable version 2.2.3 was released in November
  2017.\cite{gdalrelease}


  The OGR library was developed separately but is now a part of
  the \zk{GDAL} source tree. \zk{GDAL} used to work with raster data
  and OGR with vector data. Starting with GDAL 2.0, however, the two
  have been integrated more tightly.

  For its extensive capabilities and comprehensive set of
  functionalities, the GDAL library is widely used by both
  commercial and non-commercial \zk{GIS} projects and programs. The
  list of software programs that uses it includes Google Earth,
  ArcGIS, GRASS \zk{GIS} and many others.\cite{gdalogr}

  %% ML: Nechybi tu PostGIS?
  %% JP: Ona je uz v kapitole Teory, chtel jsem mit vsechny prostorove databaze pohromade. Pridana reference
  
  \section{PostGIS}
  
  For information on the PostgreSQL spatial database extender, refer to the section 2.2.1.

