\chapter{Practical part}
\label{4-practical}

\section{Functionality} 

The PyWPS framework allows to publish and use geoprocessing services on the server. Every process that is to be implemented by PyWPS must be constructed as a Python class, contain a list of inputs nad outputs and a handler method with two parameters - request and response. Details on the procedure of creating new processes can be found in PyWPS documentation.[2]

To send a request to PyWPS, an instance of PyWPS must be running. If it is, the request is handled and a response is generated and returned to the user. The response has a form of an XML file and includes different elemenents depending on the type of the request.


[2]http://pywps.readthedocs.io/en/master/process.html


\subsection{Delivering data to the client (data delivery?)}



\begin{itemize}

\item \textbf{Current functionality}

When the PyWPS Execute request is called, a new, temporary folder is created in (the TempPath space) and input data is moved here. While the process is being executed, there may be any number of temporary files in this folder. For every process, it should be specified in its response what the final output is. Once the execution is finished, the final output is copied to a corresponding folder (nebo location?). The temporary folder, containing input and output data and all the intermediate data that arose during the execution, is then deleted.

There are two options of delivering the output to the client to choose from. The output data can be copied directly to the  XML response. This is typically used if the output is relatively simple and (small - jak se rekne maly objem dat?). It is also the default option.

If, on the contrary, the output data is large and complex, there is another option. The client is only given a reference, a URL link, from which the data can be downloaded. In the current implementation, PyWPS saves the file in a folder specified in the configuration passed by the service (or in a default location). The URL is embedded in the XML response.[1]

It is up to the consumer of the process to decide which option to choose. For the latter option, the user must set the "@asReference" value to "True". By default, it is set to "False".

[1]http://pywps.readthedocs.io/en/master/process.html\#returning-large-data

%include diagram



\item \textbf{Proposed functionality}

The (aim of this thesis) was to develop another variant to add to the existing two that would store data in a PostGIS database (database, nebo RDBMS?). 

In this case, when an execution of a process starts, a temporary folder is created for storing the input, intermediate, and output data, just like described previously. This time, however, the output is not moved to another folder. After the final output has been produced, connection to the database is established and the output data is copied there directly (je tohle spravne???). Then, the (temporary folder) is deleted. 

When the XML response is delivered to the client, it contains a reference to the database with which the client can access the data. The reference is composed of the name of the database, schema and table. 

The only difference from the previous two options when executing a process is that another input parameter must be included in the request. The parameter, db\textunderscore section, specifies the section from which the login credentials for connecting to the database are extracted in the configuration file.

To use this functionality as a consumer of a process, it must be implemented in the process by its author. 


\end{itemize}



\section{Development} 

\begin{itemize}

\item \textbf{The PgWriter class development}

A new class, PgWriter, has been developed that implements (the proposed functionality).

PgWriter is stored in the \textbackslash pywps\textbackslash app folder. It consists of several methods that are described below. 

%co dal sem napsat?


\begin{itemize}
\item \textbf{\textunderscore \textunderscore init\textunderscore \textunderscore }

The \textunderscore \textunderscore init\textunderscore \textunderscore  method is a constructor, i.e. it is automatically called when an instance of the PgWriter class is created. 

In this method, the get\textunderscore config\textunderscore value function is used that is defined in the PyWPS API. It accesses the configuration file and retrieves required elements. What elements are retrieved is specified by the function's two parameters, section and option.

First, the name of the database is extracted from the configuration file and saved as a variable. Then, another variable is defined that serves as a connection string for connecting to a database. Requisite elements from the configuration file (user name, password and host server) are retrieved analogically. Finally, an instance of the create\textunderscore schema is created and saved to a variable.

There are three parameters passed (je passed spravne?) to the method - uuid, a unique string (da se to takhle rict?) of every process, identifier (typically the name of the process) and the section name from the corresponding configuration file.


\item \textbf{create\textunderscore schema}

This method first defines a variable schema\textunderscore name that consists of the two input parameters, separated by a dot - the identifier of the process and uuid (kde se vytvari uuid??).

Next, the psycopg2 library is used to connect to the database, specified by the "connstr" variable. A try-except clause is used to raise an exception if the connection could not be established. 

Then, when a cursor has been created, an SQL query is executed that generates a new schema if it doesn't exist already. Name of the schema is equal to the "schema\textunderscore name" variable. This, too, uses a try-except clause.

After the SQL query is performed, changes in the database are commited to make them persistent, cursor and connection are closed and the schema\textunderscore name variable is returned.


\item \textbf{store\textunderscore output}

As its name suggests, the store\textunderscore output method handles writing output data to the database. It benefits from an extensive use of the OGR library. It has two input parameters, the name of the file that is to be stored in a database, and a process identifier. 

Thanks to OGR, the process is fairly simple and straight-forward. The output file is opened using the file name input parameter and connection to the database is established. Then, data is copied from the output file to the database using the OGR CopyLayer function (co to je? method? function?). 

Each of the three above mentioned operations is followed by a simple condition that checks if the variable storing output of the operation is not None. If it is, it raises an exception with a corresponding message.



\item \textbf{store}

The purpose of this method is to wrap the functionality of the store\textunderscore output method in a loop. If a process produces multiple output files, it applies the store\textunderscore output method's functionality for each file. 

Also, a string is created that specifies the location of the data. It consists of the name of the databse, schema and identifier of the process. This string is then given to the client as an output in the XML response of the process. The only input parameter of this method is the outputs of a process.

\end{itemize}

\item \textbf{The Process class changes}

The Process class is a part of the PyWPS source code. It is the parent class to (nebo for?) all processes. If wanting to use the extension and save output data to a database, there are a few changes that must be done in this code.

A new method must be defined that is called setOutputDbStorage. This can be done at the end of the current code. This method's only input parameter (in addition to the mandatory "self" parameter) is a name of a section in the configuration file. The method creates an instance of the PgWriter class and saves it in a variable.

Obviously, names of the method and the variable are not restricted to those stated above and can be basically anything, however, these names must be used consistently throughout the process of setting up the extension.

Another snippet of code must be added in (at?) a specific location (after the response is created). It consists of a hasattr condition and a method call. The hasattr function is one of the functions built into the Python interpreter. It has two input parameters, a name of an object and a string. If the string is one of the object's attributes, it evaluates the condition as "True".[3] In this case, the object is "self" and the string is "writer". 

If the condition is met (i.e. the object - a process - has an attribute called "writer") the second command is executed and the "store" method is called.

[3] https://docs.python.org/3.6/library/functions.html\#hasattr


\item \textbf{Configuration file changes}

To connect to a remote database, the following login credentials are required - the name of the database, user name, password and a host (the server the database runs on). The PgWriter class is designed to extract them from the configuration file, so the author of a process must add a new section there that contains all the required information. 

There may be any number of sections, each can be (pointing - da se to pouzit?) to a different database. Multiple sections may only differ in user name and password while they point to the same database. The consumer of the process enters the name of the desired section as an input parameter in the request. 

\item \textbf{Changes in the process}

\begin{verbatim}
        dbsection = request.inputs['db_section'][0].data
        self.setOutputDbStorage(dbsection)
\end{verbatim}
      
\end{itemize}

