\chapter{Conclusion and future work}
\label{5-conclusion}


The aim of this thesis was to design an extension for the PyWPS
framework that would allow output data to be stored in a database
rather than in a standard file system.

Until now, there were only two ways of returning data to the
client. It was either embedded in the response directly or, typically
if the data was larger or more complex, it remained stored on a server
and the response included a reference (a~\zk{URL} link) pointing to
the location from which the data could be downloaded.

By adding the third option of storing outputs in a remote database,
the output data can be transferred, stored and processed more
effectively. However, there is a~lot of room for improvement and
future work.

Most importantly, current state of the extension provides the client
with a string that points to a specific table, schema and
database. The client, however, has to access the database by
themselves and only then work with the data.

In the future, the client should be only given a unique \zk{URL} link
%% ML: WMS pouze na okraj (jde pouze nahled o data, ale to nekdy muze
%% uzivateli take stacit), uvedte take WCS pro rastrova data
%% JP: doplneno
that points to a running \zk{WFS} (or, for viewing data only, \zk{WMS} 
and \zk{WCS}) service that will be retrieving data from the database. 
The client will not even need to be aware where tha data is stored 
as they will access it easily through a standardized interface.

The author encountered several other issues during the
development. Some of them have been solved, others were out of the
scope of this thesis.

One of the unresolved issues is safety of the database login
credentials. At this point, all the data that is neccessary for
connecting to and accessing the database (including password) is
stored in the configuration file as plain text. Obviously, this is a
major safety risk and a better, more secure solution is needed where
(at least) the password would not be accessible directly.

%% ML: velikost vystupu by mohla byt v jednoduche forme dana v
%% konfiguracnim souboru, ten aktualne obsahuje volbu pro limit
%% souboru na vstupu, ale to pouze na okraj, nechte text jak je
%% JP: ok
Another problem that may arise if using this extension and that should
be addressed in the future is exceeding the capacity of the database
if the output data that is being copied to the database is too
large. This could be solved by adding another functionality that would
establish a connection to the database, check how much space there is
available and then compare it to the size of the output file and raise
an exception if the capacity was not sufficient.

These and perhaps other improvements are neccessary before the
extension can be fully implemented by PyWPS. Due to time constraints,
the above mentioned changes will be worked on after this thesis has
been submitted. The author plans to cooperate with authors of PyWPS to
implement the extension as a pull request to the PyWPS repository.
The entire thesis, including text, source code and sample data, 
is available on GitHub at: \href{https://github.com/ctu-geoforall-lab-projects/bp-pisl-2018}{https://github.com/ctu-geoforall-lab-projects/bp-pisl-2018}.


%% ML: zminte, ze zdrojove texty BP jsou dostupne v git repozitari na
%% GitHubu vcetne URL
%% JP: pridano
