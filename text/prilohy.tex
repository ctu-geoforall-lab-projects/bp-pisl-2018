%%%%%%%%%%%%%%%%%%%%%%%%%%%%%%%%%%%%%%%%%%%%%%%%%%%%%%%%%%%%%%%%%%%%%%%%%%%%%%%%%%%
%%                 APPENDIX 1  RUN TEST.PY                                       %%
%%%%%%%%%%%%%%%%%%%%%%%%%%%%%%%%%%%%%%%%%%%%%%%%%%%%%%%%%%%%%%%%%%%%%%%%%%%%%%%%%%%
\chapter{User guide for testing} \label{testing}
\label{user-guide}

%Navigate ??
%$ cd ??

%% ML: opravil jsem soubor se zavislostmi, aby byl funkcni (1f6d2ca) a
%% cely postup jsem vyzkousel, melo by to byt v poradku, ale klidne
%% vyzkousejte tez
Using requirements.txt, go to bp-pisl-2018/src directory and install
all required packages, including PyWPS core package:
\begin{lstlisting}
$ pip3 install -r requirements.txt
\end{lstlisting}

%% ML: naklonovat repositar je prvni krok, zaroven jsem upravil URL

\noindent Clone the repository:

\begin{lstlisting}
$ git clone \
https://github.com/ctu-geoforall-lab-projects/bp-pisl-2018
\end{lstlisting}

%% ML: chybi Vam tam podstatny krok! pridani sekce db do konfigurace....
%% ML: odkazte se na prilohu A2 
\noindent Run demo application (taken from PyWPS-Demo):

\begin{lstlisting}
$ python3 demo.py
\end{lstlisting}

\noindent Run the test in another terminal - demo application must be
running concurrently:

\begin{lstlisting}
$ python3 test.py
\end{lstlisting}






%\quad \, Install Git and Python bindings for GDAL (must be installed prior to installing PyWPS):
%
%\begin{lstlisting}
%$ sudo apt install git python-gdal
%\end{lstlisting}
%
%Fetch the source code from GitHub and install PyWPS:
%
%\begin{lstlisting}
%$ sudo pip install -e git+https://github.com/geopython/pywps.git@master#egg=pywps-dev
%\end{lstlisting}
%
%Install bindings:
%
%\begin{lstlisting}
%$ sudo apt install python3-psutil 
%\end{lstlisting}




%%%%%%%%%%%%%%%%%%%%%%%%%%%%%%%%%%%%%%%%%%%%%%%%%%%%%%%%%%%%%%%%%%%%%%%%%%%%%%%%%%%
%%                 APPENDIX - USER MANUAL                                        %%
%%%%%%%%%%%%%%%%%%%%%%%%%%%%%%%%%%%%%%%%%%%%%%%%%%%%%%%%%%%%%%%%%%%%%%%%%%%%%%%%%%%
\chapter{Enabling database storage} \label{appendix}
\label{Enabling-database-storage}

To enable the database storage capacity as an author of a process,
there are a few things that must be done. No changes are neccessary in
the code of the process itself, but configuration file must be
updated.

It is assumed that there is an instance of PostGIS database running on some server.


\section{Configuration file changes} \label{cfgchanges}

When output data is not embedded directly in the response document,
there are two more options - it can be stored as a file or in a
%% ML: a value? nechybi tam neco?
database. The decision is made based on a value. Therefore, a new
option named \texttt{store\_type} must be added in the \texttt{server}
section and its value set to \texttt{db}. If the option already
exists, only the value must be changed. Position of the option within
the \texttt{server} section is arbitrary.

\begin{verbatim}
store_type = db
\end{verbatim}


%% ML: zalezi na nastaveni dabazoveho serveru (dokazu si predtavit, ze
%% bude stacit pouze nazev databaze, kdyz nebude db server na lokale
%% zapezpecen uzivatelskym heslem) v opacnem pripade jeste muze chybet
%% port, to jsou dva extremy
%% ML: zkuste vetu upravit, ted to vypada,
%% ze musi byt zadano vse, zalezi na nastaveni db serveru
To connect to a remote database, the following login credentials are
required - the name of the database, user name, password and a host
(server the database runs on). The \texttt{PgStorage} class is designed
to extract them from the configuration file, so the author must add a
new section there that contains all the required information. An
example of a complete \texttt{db} section is shown below.

\begin{verbatim} 
[db]  
host=geo102.fsv.cvut.cz
user=pisl
password=XXXXXXXX
dbname=pisl_bp
\end{verbatim}




%%%%%%%%%%%%%%%%%%%%%%%%%%%%%%%%%%%%%%%%%%%%%%%%%%%%%%%%%%%%%%%%%%%%%%%%%%%%%%%%%%%
%%                 ATTACHMENT - STRUCTURE OF (the folder)                        %%
%%%%%%%%%%%%%%%%%%%%%%%%%%%%%%%%%%%%%%%%%%%%%%%%%%%%%%%%%%%%%%%%%%%%%%%%%%%%%%%%%%%



\chapter{GitHub Repository Content}

%% ML: uvedte URL

%% ML: zip ktery budete nahravat do KOSu, tak bude v adresari text
%% obsahovat navic jeste finalni PDF prace

\label{cd}



\begin{forest}
  for tree={
    font=\ttfamily,
    grow'=0,
    child anchor=west,
    parent anchor=south,
    anchor=west,
    calign=first,
    inner xsep=10pt,
    edge path={
      \noexpand\path [draw, \forestoption{edge}]
      (!u.south west) +(7.5pt,0) |- (.child anchor) pic {folder} \forestoption{edge label};
    },
    before typesetting nodes={
      if n=1
        {insert before={[,phantom]}}
        {}
    },
    fit=band,
    before computing xy={l=25pt},
  }  
[repo
  [src
    [diff
    ]
    [processes
    ]
    [static
      [data
      ]
    ]
  ]
  [text
    [pictures
    ]
  ]
  [zadani
  ]
]
\end{forest}


