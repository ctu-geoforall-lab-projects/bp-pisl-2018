%%%%%%%%%%%%%%%%%%%%%%%%%%%%%%%%%%%%%%%%%%%%%%%%%%%%%%%%%%%%%%%%%%%%%%%%%%%%%%%%%%%
%%                 APPENDIX 1  RUN TEST.PY                                       %%
%%%%%%%%%%%%%%%%%%%%%%%%%%%%%%%%%%%%%%%%%%%%%%%%%%%%%%%%%%%%%%%%%%%%%%%%%%%%%%%%%%%
\chapter{User guide}
\label{user-guide}

%Navigate ??
%$ cd ??

Using requirements.txt, install all required packages, including PyWPS core package: 
\begin{lstlisting}
$ pip3 install -r requirements.txt
\end{lstlisting}

Clone the repository:

\begin{lstlisting}
$ git clone https://github.com/ctu-geoforall-lab-projects/bp-pisl-2018/tree/master
\end{lstlisting}

Run demo application:

\begin{lstlisting}
$ python3 demo.py
\end{lstlisting}

Run the test in another terminal (demo application must be running):

\begin{lstlisting}
$ python3 test.py
\end{lstlisting}






%\quad \, Install Git and Python bindings for GDAL (must be installed prior to installing PyWPS):
%
%\begin{lstlisting}
%$ sudo apt install git python-gdal
%\end{lstlisting}
%
%Fetch the source code from GitHub and install PyWPS:
%
%\begin{lstlisting}
%$ sudo pip install -e git+https://github.com/geopython/pywps.git@master#egg=pywps-dev
%\end{lstlisting}
%
%Install bindings:
%
%\begin{lstlisting}
%$ sudo apt install python3-psutil 
%\end{lstlisting}




%%%%%%%%%%%%%%%%%%%%%%%%%%%%%%%%%%%%%%%%%%%%%%%%%%%%%%%%%%%%%%%%%%%%%%%%%%%%%%%%%%%
%%                 PŘÍLOHA - USER MANUAL                                          %%
%%%%%%%%%%%%%%%%%%%%%%%%%%%%%%%%%%%%%%%%%%%%%%%%%%%%%%%%%%%%%%%%%%%%%%%%%%%%%%%%%%%
\chapter{User Manual}
\label{User-Manual}

To enable the database storage capacity as an author of a process, there are a few things that must be done. No changes are neccessary in the code of the process itself, but configuration file must be updated. 

It is assumed that there is an instance of PostGIS database running on some server.


\section{Configuration file changes}

When output data is not embedded directly in the response document, there are two more options - it can be stored as a file or in a database. The decision is made based on a value . Therefore, a new option named \texttt{store\_type} must be added in the \texttt{server} section and its value set to \texttt{db}. If the option already exists, only the value must be changed. Position of the option within the \texttt{server} section is arbitrary.

\begin{verbatim}
store_type = db
\end{verbatim}


To connect to a remote database, the following login credentials are required - the name of the database, user name, password and a host (server the database runs on). The \texttt{PgWriter} class is designed to extract them from the configuration file, so the author must add a new section there that contains all the required information. An example of a complete  \texttt{db} section is shown below.

\begin{verbatim}
[db]
host=geo102.fsv.cvut.cz
user=pisl
password=XXXXXXXX
dbname=pisl\_bp
\end{verbatim}




%%%%%%%%%%%%%%%%%%%%%%%%%%%%%%%%%%%%%%%%%%%%%%%%%%%%%%%%%%%%%%%%%%%%%%%%%%%%%%%%%%%
%%                 ATTACHMENT - STRUCTURE OF (the folder)                        %%
%%%%%%%%%%%%%%%%%%%%%%%%%%%%%%%%%%%%%%%%%%%%%%%%%%%%%%%%%%%%%%%%%%%%%%%%%%%%%%%%%%%



\chapter{GitHub Repository Content}
\label{cd}


\setlength{\unitlength}{.5mm}
\begin{picture}(250, 220)

  \put(  0, 212){\textbf{.}}

  \put(  1, 200){\line(0, 1){5}}
  \put(  1, 200){\line(1, 0){10} {\textbf{ src}}} 
  \put(150, 200){ Source code}  

  \put(  1,  190){\line(0, 1){10}}
  \put(  1,  190){\line(1, 0){10} {\textbf{ diff}}}
  \put(150,  190){ PyWPS source code alterations}                     
          
  \put(  1,  180){\line(0, 1){10}}
  \put(  1,  180){\line(1, 0){10} {\textbf{ text}}}
  \put(150,  180){ Text of the thesis in PDF format}
      
  \put(  1,  170){\line(0, 1){10}}
  \put(  1,  170){\line(1, 0){10} {\textbf{ zadani}}}
  \put(150,  170){ Assigned topic of the bachelor thesis}
\end{picture}